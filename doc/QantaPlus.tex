\documentclass[a4paper,12pt]{article}
\usepackage[affil-it]{authblk}
\usepackage{relsize}
\usepackage{tikz} 
\usetikzlibrary{shapes,arrows,positioning,automata,backgrounds,calc,er,patterns}
\usepackage{tikz-feynman}
\tikzfeynmanset{compat=1.0.0}
\usetikzlibrary{decorations.pathmorphing, patterns,shapes}


\usepackage[utf8]{inputenc}
\usepackage{color}
 
\usepackage{mathtools}
\usepackage{amsmath}
\usepackage{slashed}
\usepackage{calrsfs}
\usepackage{hyperref}
\usepackage{amssymb}      
\usepackage{pifont}
\usepackage{footnote}
\usepackage[toc]{appendix}
\usepackage[a4paper, total={6in, 8in}]{geometry}
%\usepackage[sc]{mathpazo} 
\usepackage{setspace}
\usepackage{systeme,mathtools}

\usepackage{braket}
\usepackage{flexisym}
\usepackage{graphicx}
\graphicspath{{images/}}
\usepackage{fancyhdr}
 
\usepackage{geometry}
 \geometry{
 	a4paper,
 	total={170mm,257mm},
 	left=20mm,
 	top=20mm,
 }
\fancyhead[R]{\bfseries} 

\newcommand{\HRule}{\rule{\linewidth}{0.3mm}}
 
\begin{document}
\begin{titlepage}
\title{\textsc{QuantaPlus} Manual and User Guide}
\author{Muhammad Hussein Muhammadahmed Ahmed}
\affil{{Department of Physics, Faculty of Sciences,\\Sudan University Of Science \& Technology Khartoum, Sudan}}
\author{Mohammed Maher Abdelrahim Mohammed}
%\affil{Dipartimento di Fisica dell'Universit\`a della Calabria \\I-87036 Arcavacata di Rende, Cosenza, Italy}
%\affil{INFN - Gruppo collegato di Cosenza \\I-87036 Arcavacata di Rende,
%	Cosenza, Italy}
\affil{{Department of Physics, Faculty of Sciences,\\Sudan University Of Science \& Technology Khartoum, Sudan}}
%\email{}
 
 
\date{\today}% It is always \today, today,
%  but any date may be explicitly specified

%\begin{abstract}
% 
%\end{abstract}
\maketitle
 
%\begin{center}
%	\line(1,0){459}
%\end{center}
\begin{abstract}
\textsc{QuantaPlus} is a modern general-purpose multi-threaded quantum computing mechanics written
in C++17 and composed solely of header files.
\end{abstract}
\end{titlepage}
 \addcontentsline{toc}{section}{Contents}
 \tableofcontents
 \begin{center}
 	 \includegraphics[scale=0.75]{QuantaPluslogo.png}
 \end{center}
\section*{Introduction}
High-performance numerical algorithms play a central role in theoretical calculations and analysis
of experimental data in (non-relativistic) quantum mechanical problems, which is the problem of predicting
the properties of systems of several quantum particles at a much larger scale~\cite{ }. The latter is
a crucial ingredient in the field of today’s computational physics due to its technological interest
and potential applications in various problems of fundamental importance in quantum chemistry,
condensed matter physics, and materials science.



\section*{Files, installation and testroutines}
The package can be downloaded from~\href{https://github.com/Moh-Maher}{github repository}.  
\subsection*{Files}
The gzipped tarred file (QuantaPlus.vvvv.tar.gz) will produce a directory QuantaPlus.vvvv with
a number of subdirectories. vvvv is version information. The created directory is called
the main directory in the remainder.
The main directory contains the files LICENSE, README.md, and a Makefile.

The subdirectory doc contains the documentation.
The subdirectory lib will after compiling contain the compiled libraries libquantaplus.a.

The subdirectory include contains all the needed header files. examples contains the testing and example programs. outputs contains the output obtained after execute some of the example files.
There are a few extra files around as well. These typically contain inputs needed or large
sets of constants.


\subsection*{Installation}

The main steps are to run make in the main directory. This should produce the files
libquantaplus.a and also send a copy to the lib subdirectory. You might
have to change the variables CC, CFLAGS and CFLAGTESTS. CC should specify the C++
compiler and the options to be used for everything. CFLAGS can be used to specify addi-
tional options in compiling the libraries and CFLAGTESTS to specify additional options for
the testing programs.
“make clean” can be used to remove many of the files created during compiling.
The actual installation is by putting the contents of the include directory somewhere
in the include path of your compiler and the two files libjbnumlib.a and libchiron.a
somewhere in the library path. For many C++ compilers the paths are given in the
environment variables CPLUS INCLUDE PATH and LIBRARY PATH respectively.
\subsection*{testroutines}

For every file xxx.h included for QuantaPlus there is a testing/example code
testxxx.cpp in the subdirectory test. These can be compiled using “make testxxx” in the
main directory. Executing the resulting file QUANTA.out should then produce output identical (up
to the precision specified and possible randomly generated cases) to the file testxxx.dat
in the subdirectory testoutputs.
  
\end{document}


